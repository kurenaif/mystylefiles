\documentclass[12pt]{article}
\usepackage{mystyle}
\usepackage{cases}

\addbibresource{ref.bib}

\begin{document}
\section{滑面壁モデル}
壁面隣接セル内で, 壁面へ行こう方向に近似した運動量式
\begin{equation*}
 \frac{\partial}{\partial y^*} \left( (\mu + \mu_t)\ \frac{\partial U}{\partial y^*} \right) = \frac{\nu^2}{k_P} \left( \frac{\partial}{\partial x}\ (\rho U U) + \frac{\partial P}{\partial x} \right)
\end{equation*}
この式の右辺を$C_U$とし, 
\begin{equation}
 \label{eq:momentum}
 \frac{\partial}{\partial y^*} \left( (\mu + \mu_t)\ \frac{\partial U}{\partial y^*} \right) = C_U
\end{equation}
ここで, $y^* \equiv y \sqrt{k_p} / \nu $
乱流渦粘性係数$\mu_t$を, 1方程式モデルのように近似
\begin{equation}
 \mu_t = \rho c_\mu k^{1/2} l = \rho c_\mu k^{1/2} c_l y \simeq \alpha \mu y^*
\end{equation}
ここで, $\alpha = c_l c_\mu$ とモデル化する.
さらに, 粘性低層を考慮して, 
\begin{equation}
  \mu_t = \max(0, \alpha \mu (y^* - y_v^*))
\end{equation}
式\ref{eq:momentum}を解析的に積分することが可能. 粘性低層厚さ$y_v^*$を境にして
\begin{subnumcases}
 {\frac{\partial U}{\partial y^*} = }
 \label{eq:partial_u_a} (C_U y^* + A_U)/\mu, \mathrm{if\ } y^* < y_v^* & \\ 
 \label{eq:partial_u_b} \frac{C_U y^* + A'_U}{\mu(1 + \alpha (y^* - y_v^*))}, \mathrm{if\ } y^* \geq y_v^* &
\end{subnumcases}
これを更に積分すると
\begin{subnumcases}
 {U = }
  \label{eq:u_a} \frac{1}{2 \mu} C_U {y^*}^2 + \frac{1}{\mu} A_U y^* + B_U, \mathrm{if\ } y^* < y_v^*& \\
  \label{eq:u_b} \frac{C_U}{\alpha \mu} + \left( \frac{A_U'}{\alpha \mu} - \frac{C_U}{\alpha^2 \mu} (1 - \alpha y_v^*) \right) \ln(1 + \alpha(y^* - y_v^*)) + B_U', \mathrm{if\ } y^* \geq y_v^* &
\end{subnumcases}
これに以下の境界条件を導入する. これを用いて, 定数 $A_U, B_U, A_U', B_U'$を求める.

\begin{itemize}
 \item $u|_{y=0} = 0$
 \item $u|_{y=y_n} = u_n$
 \item $y^* = y_v^*$ で 連続条件
\end{itemize}

\end{document}
